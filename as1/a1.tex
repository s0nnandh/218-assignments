\documentclass{extarticle}
\usepackage[a4paper,left=1in,right=1in,bottom=1in]{geometry}
\usepackage[parfill]{parskip}
\usepackage{ragged2e}
\usepackage{amsmath} 
\usepackage{amsthm} 
\usepackage{graphics}
\usepackage{amssymb}
\usepackage{upgreek}
\usepackage{listings}
\usepackage{esint}
\usepackage{floatrow}
\newtheorem{theorem}{Theorem}
\newtheorem{lemma}{Lemma}
\newtheorem{corollary} [theorem] {Corollary}
\newtheorem{proposition}{Proposition}[section]
\theoremstyle{remark}
\newtheorem*{remark}{Remark}
\usepackage[ruled,linesnumbered,vlined]{algorithm2e}
\usepackage{xcolor}
\usepackage{listings}
\usepackage{color}
\definecolor{name}{rgb}{0.5,0.5,0.5}
\usepackage{hyperref}
\usepackage{url}
\usepackage{multirow}
\usepackage{fancyhdr}
\usepackage{graphicx}
\newcommand{\y}[1]{\mathcal{O}(#1)}
\pagestyle{fancy}
\fancyhf{}
%\setlength{\parindent}{7em} 
\lhead{190050043-190050055-190050077-190050079}
\rhead{Assignment-1}
\cfoot{Page \thepage}
\renewcommand{\footrulewidth}{1pt}
\newcommand{\tbf}[1]{\textbf{#1}}
\setcounter{tocdepth}{2}
\usepackage[utf8]{inputenc}
\begin{document}
\title{Assignment-1}
\author{190050043,190050055,190050077,1900500079}
\date{January, 2021}
\maketitle
%\tableofcontents
\thispagestyle{empty}
\clearpage
\pagenumbering{arabic}

\sffamily

\begin{algorithm}[H]
    \SetAlgoLined
    \KwData{Graph,source}
    \KwResult{Distances of each vertex from the source and starting edges of the shortest paths}
    create priority queue Q of vertices of Graph //min-heap\\
    \ForEach {vertex v in Graph}{
    $dist[v] \gets \infty$\\
    $prev[v] \gets \text{UNDEFINED}$\\
    $\text{add} v \text{to} Q$ 
    }
    $dist[source] \gets 0$\\
    \While{Q is not empty do}{
    $u \gets Q.extract()$\\
    \ForEach{neighbour v $\in$ Q of u}{
    $alt \gets dist[u]+length(u,v)$\\
    \If{alt $<$ dist[v]}{
    $dist[v] \gets alt$\\
    $prev[v] \gets u$
    }
    }
    }
    $\text{return} ~dist[],prev[]$
    \caption{Dijkstra's Algorithm}
    \end{algorithm}
%\LARGE{Question1}\\

\Large{Question 2}\\
\normalsize


Subset b) is the required optimal solution for the rest of the cases we can come up
with a counter example.

%\LARGE{Question3}\\

\Large{Question4}\\
\normalsize

\tbf{a)}\\

We are required to find a minimum weight subset of edges to be removed so that there are no cycles in G.
We can rephrase this question to find the \tbf{Maximum Spanning Tree} because at the end we want
a graph which is connected and has no cycles which essentially means that it is a tree.
But we are required to remove smaller cost edges so the tree with which we obtain at last 
will be a \tbf{Maximum Spanning Tree}.\\
In order to solve this problem we can follow the algorithm for finding the Minimum spanning tree
as discussed in the class with the help of a trick such that since all the edges are positive
and not equal to zero what we can do is we can convert the cost values to their
additive inverses i.e, take every cost value and multiply by -1.\\
Now for this graph with all costs of edges find it's minimum spanning tree as discussed in 
class either by \tbf{Kruskal's Algorithm} or \tbf{Prim's Algorithm}. After finding the minimum tree for 
this graph multiply all the edges in the MST with -1 and remove this edges from the 
set of edges given at the start of the problem the set of remaining edges is the reuqired set.\\

\tbf{b)}\\

In this problem we are given a graph and a set of vertices. Let this set of vertices be $R$
and it's complement set of vertices be $R^{c}$.Now we are required to connect every vertex in
$R^{c}$ to $R$ and the set of edges that are required to do this must be minimum.In order 
to solve this problem we can do this greedily and this is similiar to \tbf{Prim's Algorithm}.\\
So every time when we are required to add a edge from $R^{c}$ to $R$ we consider all the edges 
that are coming out from $R$ to $R^{c}$ and we greedily select the edge which has the least 
weight and we add that edge to our reuired subset and we add the corresponding vertex from the 
edge which is in $R^{c}$ to $R$ and go on until $R^{c}$ is empty i.e, all the edges which 
were not in $R$ are now in the set .




\end{document}
 